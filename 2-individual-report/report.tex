\documentclass{beamer}

\usepackage[italian]{babel}
\usepackage{array}
\usepackage{hyperref}
\usepackage[style=mla]{biblatex}
\addbibresource{./bibliography.bib}

\usecolortheme{fly}
\setbeamertemplate{navigation symbols}{}

\title{
	Applicazione di metodi formali \\
	\small Sintesi di \cite{sobel}
}
\author{Stefano Volpe}
\institute{Università di Bologna}
\date{\today}
\logo{\includegraphics[width=0.1\textwidth]{by-nc-sa-4-0}}

\begin{document}

\begin{frame}
	\titlepage
\end{frame}

\begin{frame}{Indice}
	\tableofcontents
\end{frame}

\section{Introduzione}
\begin{frame}{Introduzione}
	Due classi di studenti in Ingegneria del software all'Università di Miami
	svolgono lo stesso progetto in piccoli gruppi:
	\begin{itemize}
		\item i gruppi ``di controllo`` (C) fanno uso di metodi di sviluppo
		      tradizionali;
		\item i gruppi ``di trattamento`` (T) hanno studiato i \textbf{metodi
			      formali per lo sviluppo di programmi}.
	\end{itemize}
\end{frame}

\section{Approccio sperimentale}
\begin{frame}{Approccio sperimentale}
	\begin{table}
		\begin{tabular}{|p{0.5\linewidth}|c|c|}
			\hline
			                                            & C  & T  \\
			\hline
			Introduzione alla derivavione dei programmi & No & Sì \\
			\hline
			Semantica delle strutture di dati           & Sì & Sì \\
			\hline
			Progettazione orientata agli oggetti        & Sì & Sì \\
			\hline
			Analisi formale di programmi concorrenti    & No & Sì \\
			\hline
			Ingegneria del software                     & Si & Sì \\
			\hline
			Laboratorio di sviluppo software            & Si & Sì \\
			\hline
		\end{tabular}
		\caption{al di là della formazione specifica sui metodi formali, i due
			gruppi condividono lo stesso curriculum.}
	\end{table}

\end{frame}

\section{Il progetto}
\begin{frame}{Il progetto}
	Ogni gruppo deve realizzare un \textbf{sistema incorporato} per la gestione
	delle richieste effettuate nei confronti di \textbf{un ascensore}. È richiesto
	l'uso:
	\begin{itemize}
		\item dei principi della progettazione orientata agli oggetti;
		\item di \texttt{C++} per l'implementazione;
		\item di MFC per l'interfaccia grafica.
	\end{itemize}
\end{frame}

\section{Progettazione}

\subsection{Gruppi di controllo}
\begin{frame}{Progettazione: gruppi di controllo}
	Nessun gruppo ha consegnato diagrammi UML.
	\\~\\
	Tredici gruppi hanno consegnato prototipi. Di questi, nove hanno esibito
	codice sorgente.
	\\~\\
	Tutti i progetti sorgente consegnati mostrano un \textbf{grado di
		accoppiamento estremamente alto}: il sistema dell'ascensore e la gestione
	delle schermate convivono negli stessi moduli.
\end{frame}

\subsection{Gruppi di trattamento}
\begin{frame}{Progettazione: gruppi di trattamento}
	Tre gruppi su sei hanno consegnato diagrammi UML; di questi, uno solo presenta
	un alto grado di accoppiamento.
	\\~\\
	Quattro gruppi su sei presentano specifiche formali complete. Ad esempio:
	\begin{exampleblock}{Postcondizione per ogni operazione dell'ascensore}
		\[
			\begin{array}{l}
				(\exists p : Person \mid OnFloor(e, p)                                               \\
				\wedge \quad e.direction == p.direction : AddPerson(p))                              \\
				\wedge \quad (GoingDown(e) \rightarrow e.current\_{}floow := e.current\_{}floor - 1) \\
				\vee \quad GoingUp(e) \rightarrow e.current\_{}floor := e.current\_{}floor + 1)      \\
				\wedge \quad (\exists p : Person \mid p.ending\_{}floor ==                           \\
				e.current\_{}floor : RemovePerson(p)).
			\end{array}
		\]
	\end{exampleblock}
\end{frame}

\section{Implementazione}

\subsection{Gruppi di controllo}
\begin{frame}{Implementazione: gruppi di controllo}

\end{frame}

\subsection{Gruppi di trattamento}
\begin{frame}{Implementazione: gruppi di trattamento}

\end{frame}

\section{Una soluzione interamente formale}

\subsection{Specifiche}
\begin{frame}{Soluzione formale: specifiche}

\end{frame}

\subsection{Progettazione}
\begin{frame}{Soluzione formale: progettazione}

\end{frame}

\subsection{Implementazione}
\begin{frame}{Soluzione formale: implementazione}

\end{frame}

\subsection{Confronto con soluzioni già pubblicate}
\begin{frame}{Soluzione formale: confronto}

\end{frame}

\section{Conclusioni}
\begin{frame}{Conclusioni}

\end{frame}

\section{Sommario}
\begin{frame}{Sommario}

\end{frame}

\begin{frame}{Bibliografia}
	\printbibliography
\end{frame}

\end{document}
